% !TEX root = ../Ausarbeitung.tex
\newpage
\section{Aktuelle Lage}
\label{sec:AktuelleLage}
\Abbildung{Stats1} zeigt, dass trotz einiger Risiken der Container Technologie Unternehmen weltweit immer mehr Geld in die Containerisierung ihres Unternehmens investieren. Laut einer Umfrage, welche auf der DockerCon durchgeführt wurde, investierten 32\% der Unternehmen mindestens 500.000\$ jährlich, um die Containerisierung in ihrer Organisation voranzutreiben. \citep{Investments}
\begin{figure}[H]
	\begin{center}
		\includegraphics[width=0.8\textwidth]{ContainerInv.png}
	\end{center}
	\caption[Investitionen in die Containerisierung]{Investitionen in die Containerisierung}
	\label{fig:Stats1}
\end{figure}

\newpage
Aus \Abbildung{Stats2} ist zu entnehmen, dass 83\% der produktiv eingesetzten Container von Docker stammen. An zweiter Stelle der meist verwendeten Container findet sich CoreOS, welches von der Firma Red Hat übernommen wurde. Mesos Containerizer und \ac{LXC} machen zusammen nur 5\% aller eingesetzten Container aus. \citep{stats}
\begin{figure}[H]
	\begin{center}
		\includegraphics[width=1\textwidth]{DockerProzent.png}
	\end{center}
	\caption[Benutzung von Containertechnologien]{Benutzung von Containertechnologien \footnotemark}
	\label{fig:Stats2}
\end{figure}
\quellefoot{https://www.dailyhostnews.com/wp-content/uploads/2018/05/d2.png}
Laut eines Berichts der Container-Monitoring-Firma Sysdig stieg die Anzahl der durchschnittlich verwendeten Container pro Host 2018 um 50\% im Vergleich zum Vorjahr. Das entspricht nun etwa 15 Containern. Laut des Berichts ist 154 die höchste Anzahl von Containern, die bisher gleichzeitig auf einer Maschine laufen. \citep{stats} \Abbildung{Stats3} stellt dies grafisch dar.
\begin{figure}[H]
	\begin{center}
		\includegraphics[width=0.8\textwidth]{ContainerHost.png}
	\end{center}
	\caption[Container je Maschine]{Container je Maschine \footnotemark}
	\label{fig:Stats3}
\end{figure}
\quellefoot{https://www.dailyhostnews.com/wp-content/uploads/2018/05/d1.png}
\newpage
Kubernetes sei die meist genutzte Plattform, um Container zu orchestrieren und wird von Software-Unternehmen wie Microsoft und IBM verwendet. Das beliebteste Tool, um Container-Cluster für große Firmen auszurollen, sei jedoch Mesos Containerizer. \citep{stats} Die genaue Verteilung ist \Abbildung{Stats4} zu entnehmen.
\begin{figure}[H]
	\begin{center}
		\includegraphics[width=0.8\textwidth]{Manager.png}
	\end{center}
	\caption[Cluster-Manager]{Cluster-Manager \footnotemark}
	\label{fig:Stats4}
\end{figure}
\quellefoot{https://www.dailyhostnews.com/wp-content/uploads/2018/05/d4.png}

